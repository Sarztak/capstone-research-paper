% Chapter: Conclusion
\section{Proposed Conclusion}

This study is expected to contribute both a conceptual framework and a working prototype that clarify how large language models (LLMs) can participate meaningfully in the design of drone light shows. Rather than seeking to prove autonomous generation or full automation, the project is positioned to reveal the structural relationships that enable effective collaboration between semantic creativity and syntactic enforcement.

We anticipate that the implemented pipeline will illustrate how LLMs, when treated as semantic front ends, can provide a fluid and intuitive interface for expressing creative intent. At the same time, the integration of analytic solvers and the \textbf{Skybrush Studio API} is expected to demonstrate that operational safety, trajectory validity, and physical feasibility must remain grounded in formal systems external to the model. In this configuration, the language model proposes and interprets, the analytic subsystem verifies and refines, and the human designer remains in control of the overall aesthetic and conceptual direction.

The broader implication of this work lies in its methodological stance. It suggests that progress in creative robotics does not depend on the expansion of model capability alone, but on the thoughtful coupling of generative and analytical systems through transparent interfaces. This hybrid architecture—semantic at the front, syntactic at the core—may provide a reproducible pattern for other creative domains where expressive design must coexist with engineering constraints.

Looking ahead, we expect that future iterations of this research could:
\begin{itemize}
  \item Extend the current pipeline to incorporate real-time feedback and adaptive re-optimization during flight simulation or preview.
  \item Develop a fully integrated interface between the LLM and the Skybrush Studio API using principles derived from the Model Context Protocol (MCP), enabling direct voice or text-based editing within design software such as Blender.
  \item Evaluate the system in collaboration with professional show designers to assess usability, efficiency, and creative satisfaction.
\end{itemize}

If successful, these future stages would move the system closer to a genuinely co-creative design environment—one that unites the interpretive richness of language, the rigor of analytical validation, and the creative intuition of human artists. The anticipated contribution of this project, therefore, is not a complete solution, but a framework for integration: a pathway toward balancing semantic imagination with syntactic precision in the design of drone light shows and, by extension, in other emerging forms of computational creativity.
