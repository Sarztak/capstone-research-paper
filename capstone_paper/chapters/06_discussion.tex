% Chapter: Discussion
\section{Discussion}

The development of this prototype clarifies both the possibilities and the limits of linking generative inputs with analytical validation in the context of drone light show design. Although the system is intentionally constrained to 2D silhouettes and static formations, its construction reveals several broader themes relevant to hybrid creative-analytic workflows.

\subsection{Separation of Creative Input and Analytical Enforcement}

A central observation is that separating semantic input from syntactic enforcement simplifies integration with production-grade tools. By treating the input image as a high-level expression of intent and delegating feasibility checks entirely to Skybrush Studio, the pipeline avoids the brittleness typically associated with end-to-end generative systems. This division of labor proved practical: transformations remained interpretable, and failures were easier to diagnose because each stage preserved a clear functional boundary.

\subsection{Role of Sampling in Bridging Representation Layers}

The sampling stage emerged as the central intermediary between conceptual imagery and executable formation data. Converting a silhouette into a discrete set of drone positions requires balancing visual fidelity with spatial constraints, and the behavior of different sampling strategies directly influenced that balance. Grid sampling maintained structural alignment but produced uneven point density; blue-noise sampling created visually dispersed patterns with reduced recognizability; and farthest-point sampling provided the most consistent preservation of overall form while maintaining minimum separation between points. These observations indicate that sampling strategy, rather than the generative origin of the image, plays a determining role in the quality and feasibility of early-stage formation design.

\subsection{Practical Interoperability with Production Tools}

Importing sampled coordinates into Skybrush Studio validated the practical feasibility of the pipeline. The tool's existing infrastructure for visualization, scaling, and collision checking absorbed the upstream variability of the sampling process without modification. This reinforces the value of designing systems that complement, rather than replace, established workflows. The pipeline demonstrates that lightweight generative front ends can integrate meaningfully with industrial solvers even when the generative component is minimal.

\subsection{Limitations of the Current Approach}

The prototype exposes clear limitations. The exclusive focus on 2D silhouettes constrains the expressive range of formations and precludes volumetric or dynamic choreography. The absence of temporal modeling means that the system cannot produce transitions or full sequences. Furthermore, the reliance on external image generation limits semantic nuance: an image captures shape but not narrative or motion. These constraints reflect scope rather than failure, but they delimit the system's usefulness for full production pipelines.

\subsection{Implications for Future Work}

Despite its limitations, the prototype illustrates how a generative-analytic pipeline can be constructed incrementally. Future extensions may incorporate 3D mesh sampling, temporal optimization, or deeper integration with language-driven interfaces. More broadly, the project suggests that hybrid systems in creative robotics benefit from modular architectures in which generative tools provide flexible inputs while analytical solvers ensure feasibility. Such an approach may offer a viable template for other domains where expressive design must coexist with formal operational constraints.

