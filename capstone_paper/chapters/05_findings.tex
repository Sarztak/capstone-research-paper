% Chapter: Findings
\section{Findings}

This project does not aim to demonstrate autonomous LLM-driven choreography, but rather to assess the feasibility of a constrained pipeline that links generative visual input to analytically validated drone formations. The findings presented here reflect the empirical performance of the implemented system and the practical constraints observed during its construction.

\subsection{1. Sampling Quality and Formation Fidelity}

Evaluation of the image-to-formation stage demonstrated that sampling strategy plays a decisive role in the recognizability and structural coherence of the resulting drone layouts. Grid sampling proved efficient but produced uneven density. Blue-noise sampling yielded visually scattered formations. Farthest-point sampling consistently preserved global structure while maintaining minimum separation constraints, making it the most suitable method for converting silhouettes into stable drone formations.

\subsection{2. Integration Feasibility with Skybrush Studio}

The pipeline successfully exported sampled coordinates in \texttt{.csv} format and imported them into Skybrush Studio via the Blender extension. This demonstrated that simple, image-derived formations can be validated and visualized using professional-grade tooling without modification to existing workflows. The import process also highlighted the importance of coordinate normalization and consistent scaling to ensure accurate spatial interpretation within Skybrush.

\subsection{3. Constraints of 2D Static Formations}

Because the implemented system focuses on static 2D silhouettes, it does not address volumetric formations, temporal transitions, or dynamic choreography. This constraint clarified the limits of using images as semantic inputs: while effective for capturing shapes, they do not encode motion, timing, or 3D structure. These limitations reinforce the need for additional stages if such a pipeline were extended toward full performance design.

\subsection{4. Practical Insights for Pipeline Design}

The development process yielded several practical insights regarding modular system design. First, separating semantic input from analytical validation reduces failure modes when integrating heterogeneous tools. Second, using established solvers such as Skybrush Studio for constraint enforcement ensures that prototype systems remain compatible with real production infrastructures. Third, even limited generative input—such as silhouettes—can meaningfully accelerate early-stage formation design when paired with deterministic post-processing.

\subsection{5. Summary}

Overall, the findings indicate that a constrained, image-driven pipeline can produce analytically valid drone formations and interface cleanly with existing professional tooling. While limited in scope, the prototype demonstrates a practical method for linking generative visual concepts to executable formation data, and provides a foundation for future extensions into 3D sampling, temporal planning, or deeper integration with language-driven interfaces.

